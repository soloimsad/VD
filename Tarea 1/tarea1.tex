\documentclass[12pt, a4paper]{article}
\usepackage[spanish]{babel}
\usepackage[utf8]{inputenc}
\usepackage{graphicx}
\usepackage{geometry}
\usepackage{fancyhdr}
\usepackage{float}
\usepackage{titling}
\usepackage{hyperref}
\usepackage{url}

% Márgenes
\geometry{a4paper, margin=2.5cm}

% Encabezado y pie de página
\pagestyle{fancy}
\fancyhf{}
\rhead{\includegraphics[height=1.2cm]{images/logo-usm.png}}
\lhead{Grupo 19\\Visualización de Datos}
\rfoot{Página \thepage}

% Configuración del logo en portada
\pretitle{
  \begin{center}
  \vspace{1cm}
  \includegraphics[width=0.5\textwidth]{images/logo-usm.png}\\
  \vspace{1.5cm}
  \LARGE
}
\posttitle{\end{center}}

% Título del informe
\title{Informe: Tecnología en la Vida Cotidiana}
\author{Felipe Campaña, Javier Gómez, Matias Elgueta}
\date{\today\\[2cm]}

\begin{document}
\maketitle

% ---------------------------------------------------------------------------------
\section*{Criterios de Selección}
\begin{itemize}
    \item Criterio 1: Porcentaje de uso de internet en el mundo
    \item Criterio 2: Porcentajes de contratación de fibra en el mundo (Top 10)
    \item Criterio 3: Horas en redes sociales por país
    \item Criterio 4: Evolución de clientes de Telecomunicación
    \item Criterio 5: **
    \item Criterio 6: **
\end{itemize}

% ---------------------------------------------------------------------------------
\section*{Análisis por Integrante}

% ===================== FELIPE CAMPAÑA =====================
\subsection*{Integrante 1: Felipe Campaña}

\subsubsection*{Criterios Seleccionados}
\begin{itemize}
    \item Porcentaje de uso de internet en el mundo.
    \item Porcentajes de contratación de fibra en el mundo (Top 10).
\end{itemize}

\subsubsection*{Justificación: Contratación de Fibra Óptica Fija}
Este indicador representa cuántas personas por cada 100 habitantes tienen acceso a Internet mediante conexiones de alta velocidad y calidad. 

\begin{itemize}
    \item Evalúa el nivel de infraestructura tecnológica en cada país.
    \item Refleja el grado de modernización digital más allá de la simple conectividad.
    \item Relacionado con la capacidad de ofrecer servicios como streaming, teletrabajo, etc.
\end{itemize}

\subsubsection*{Justificación: Acceso a Internet en la Población}
Este indicador señala el porcentaje de personas que utilizan Internet, sin importar el tipo de conexión.

\begin{itemize}
    \item Entrega una visión inclusiva del uso digital básico en cada país.
    \item Identifica regiones con barreras fundamentales de conectividad.
    \item Refleja impacto de políticas públicas y accesibilidad económica.
\end{itemize}

\subsubsection*{Gráfico 1: Uso de internet de las personas en el mundo}
\begin{figure}[H]
    \centering
    \includegraphics[width=0.85\textwidth]{images/Grafico_uso_de_internet_FC.png}
    \caption[1]{Fuente: Elaboración propia con datos de \href{https://data.worldbank.org}{World Bank} 
    (\url{https://data.worldbank.org/indicator/IT.NET.USER.ZS}).}
\end{figure}

\textbf{Conclusión:}
\begin{itemize}
    \item Europa, Oceanía y partes de Asia y América superan el 80\% de cobertura.
    \item África central y el sudeste asiático presentan niveles muy bajos (<40\%).
    \item Se visualizan claramente desigualdades tecnológicas globales.
\end{itemize}

\subsubsection*{Gráfico 2: Top 10 países con mayor porcentaje de fibra contratada}
\begin{figure}[H]
    \centering
    \includegraphics[width=1\textwidth]{images/Grafico_fibra_contratada_FC2.png}
    \caption[2]{Fuente: Elaboración propia con datos de \href{https://data.worldbank.org}{World Bank} 
    (\url{https://data.worldbank.org/indicator/IT.NET.BBND.P2}).}
\end{figure}

\textbf{Conclusión:}
\begin{itemize}
    \item Países como Mónaco, Andorra y Liechtenstein lideran con más del 50\%.
    \item La mayoría son países pequeños con economías desarrolladas.
    \item Muestra diferencias internas incluso entre regiones avanzadas.
\end{itemize}

% ===================== JAVIER GÓMEZ =====================
\subsubsection*{Integrante 2: Javier Gómez}
Los criterios seleccionados por este integrante son los siguientes:
\begin{itemize}
    \item Criterio 1: Horas diarias en redes sociales.
    \item Criterio 2: Contratación de Telecomunicaciones.
\end{itemize}

\subsection{Justificación de Horas diarias en Redes Sociales}

Puede otorgar segmentación de audiencias, esto significa identificar que población tiene alto uso en redes sociales para lanzar campañas publicitarias en distintas redes sociales. \\
Desde otro ángulo podemos realizar un estudio si existe alguna relación entre la cantidad de horas diarias en el uso de \boldmath{RRSS} frente al rendimiento de estudiantes o salud mental de una persona.

\begin{itemize}
    \item Si estudiantes en un país pasan cierta cantidad de horas diarias en las redes, las instituciones educativas podrían implementar talleres sobre gestión del tiempo.
    \item Realizar estudio frente si existe relación entre las redes sociales y trastornos sicológicos.
\end{itemize}

\subsection{Justificación Contratación de Telecomunicaciones}
Revela qué compañías dominan la provisión de internet, destacando posibles monopolios u oligopolios, ej: Claro con el 45\% de contrataciones, lo que permite evaluar competencia real y calidad del servicio.

\begin{itemize}
    \item Muestra si los consumidores priorizan precios bajos, velocidad, cobertura u otros factores al elegir un proveedor
\end{itemize}
\subsection*{Gráfico 3: Horas diarias en redes sociales por país}
\begin{figure}[H]
    \centering
    \includegraphics[width=0.85\textwidth]{images/graph1_JG.png}
    \caption{
        Fuente: Elaboración propia con datos de Statista (2024). 
        \textit{Promedio de minutos diarios de uso de redes sociales por los internautas en países seleccionados en el tercer trimestre de 2023}. 
        Recuperado de \url{https://www.statista.com/statistics/270229/usage-duration-of-social-networks-by-country/}
    }
\end{figure}


\subsubsection*{Conclusión}
Texto de conclusión específico para este gráfico. Analizar tendencias observadas y su relación con el impacto tecnológico en la vida cotidiana.

\subsection*{Gráfico 4: Evolución clientes por compañía de Telecomunicación}
\begin{figure}[H]
    \centering
    \includegraphics[width=0.85\textwidth]{images/graph2_JG.png}
    \caption{
        Fuente: Elaboración propia con datos de Subsecretaría de Telecomunicaciones (2025), 
        \textit{Informe del Sector Telecomunicaciones: Cierre 2024}, p. 13. 
        Disponible en \url{https://www.subtel.gob.cl/wp-content/uploads/2025/02/Informe_del_Sector_Telecomunicaciones_Dic24.pdf}
    }
\end{figure}

\subsubsection*{Conclusión}
Texto de conclusión específico para este gráfico. Analizar tendencias observadas y su relación con el impacto tecnológico en la vida cotidiana.



% ===================== MATÍAS ELGUETA (opcional) =====================
% Aquí puedes incluir una subsección para Matías Elgueta si va a desarrollar criterios o gráficos

% ---------------------------------------------------------------------------------
\section*{Conclusiones Generales}
\begin{itemize}
    \item Opcional**
\end{itemize}

\end{document}
