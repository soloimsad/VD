\documentclass[12pt, a4paper]{article}
\usepackage[spanish]{babel}
\usepackage[utf8]{inputenc}
\usepackage{graphicx}
\usepackage{geometry}
\usepackage{fancyhdr}
\usepackage{float}
\usepackage{titling}
\usepackage{hyperref}
\usepackage{url}




% Márgenes
\geometry{a4paper, margin=2.5cm}

% Encabezado y pie de página
\pagestyle{fancy}
\fancyhf{}
\rhead{\includegraphics[height=1.2cm]{images/logo-usm.png}}
\lhead{Grupo 19\\Visualización de Datos}
\setlength{\headheight}{20pt}
\setlength{\headsep}{1cm}
\rfoot{Página \thepage}

% Configuración del logo en portada
\pretitle{
  \begin{center}
  \vspace{1cm}
  \includegraphics[width=0.5\textwidth]{images/logo-usm.png}\\
  \vspace{1.5cm}
  \LARGE
}
\posttitle{\end{center}}

% Título del informe
\title{Percepciones y Uso de la Inteligencia Artificial}
\author{Felipe Campaña, Javier Gómez, Matias Elgueta}
\date{\today\\[2cm]}

\begin{document}
\maketitle

% ---------------------------------------------------------------------------------
\vspace*{0.3cm}
\begin{figure}[H]
    \centering
    \begin{minipage}[t]{0.45\linewidth}
        \centering
        \includegraphics[width=\linewidth]{Graficos/1.png}
        \caption{Primera parte de la infografía: uso, confianza y preocupación}
    \end{minipage}
    \hfill
    \begin{minipage}[t]{0.45\linewidth}
        \centering
        \includegraphics[width=\linewidth]{Graficos/2.png}
        \caption{Segunda parte de la infografía: cruces de variables y percepciones combinadas}
    \end{minipage}
\end{figure}



\section*{Criterios de Selección}
\begin{itemize}
    \item Criterio 1: Frecuencia de uso de IA
    \item Criterio 2: Principales usos de IA	
    \item Criterio 3: Confianza en IA
    \item Criterio 4: Frecuencia vs Confianza
    \item Criterio 5: Preocupación laboral
    \item Criterio 6: Uso principal vs Preocupación laboral	
\end{itemize}

% ---------------------------------------------------------------------------------
\section*{Análisis por Integrante}

% ===================== FELIPE CAMPAÑA =====================
\subsection*{Integrante 1: Felipe Campaña}

\subsubsection*{Criterios Seleccionados}
\begin{itemize}
    \item Frecuencia de uso de IA.
    \item Principales usos de IA.
\end{itemize}

\subsubsection*{Justificación: Frecuencia de uso de IA.}
Este indicador representa la regularidad con la que las personas usan herramientas con inteligencia artificial en su vida diaria, desde un uso esporádico hasta un uso intensivo.

\begin{itemize}
    \item Permite medir el nivel de exposición tecnológica y familiaridad de los usuarios con la IA.
    \item Da cuenta del grado de integración de estas herramientas en rutinas personales o académicas.
    \item Es útil para interpretar otras variables como confianza, percepción o utilidad de la IA.
\end{itemize}

\subsubsection*{Justificación: Principales usos de IA.}
Este indicador muestra los fines más comunes para los cuales se utiliza la inteligencia artificial, como estudiar, crear arte, buscar ideas o programar.

\begin{itemize}
    \item Ayuda a identificar el enfoque funcional que los usuarios dan a la IA en su vida cotidiana.
    \item Refleja cómo las personas aprovechan estas herramientas en contextos académicos, creativos o recreativos.
    \item Es clave para entender el tipo de valor que los usuarios perciben en estas tecnologías.
\end{itemize}


\subsubsection*{Gráfico 1: Frecuencia de uso de IA.}
\begin{figure}[H]
    \centering
    \includegraphics[width=0.85\textwidth]{Graficos/Radar_frec_ia_FC.png}
    \caption[1]{Fuente: Elaboración propia con datos}
\end{figure}

\textbf{Conclusión:}
\begin{itemize}
    \item La mayoría de los encuestados utiliza la IA principalmente para estudiar (65\% del total).
    \item Los usos relacionados con programación (17\%) y búsqueda de ideas (10\%) ocupan el segundo y tercer lugar.
    \item El uso recreativo o de entretención representa una minoría (7\%), lo que indica un enfoque más académico o productivo.
    \item El gráfico evidencia que la IA es percibida como una herramienta funcional más que como una distracción.
\end{itemize}

\subsubsection*{Gráfico 2: Principales usos de IA.}
\begin{figure}[H]
    \centering
    \includegraphics[width=0.85\textwidth]{Graficos/Treemap_Uso_IA_FC.png}
    \caption[2]{Fuente: Elaboración propia con datos}
\end{figure}

\textbf{Conclusión:}
\begin{itemize}
    \item El uso diario de IA destaca significativamente, representando la categoría con mayor frecuencia entre los encuestados.
    \item El uso semanal también es relevante, aunque menor en comparación con el uso diario.
    \item El gráfico sugiere una integración intensiva de la IA en las rutinas diarias de los participantes.
    \item Esto puede reflejar tanto una alta dependencia como una adaptación natural al uso frecuente de estas tecnologías.
\end{itemize}


% ===================== JAVIER GÓMEZ =====================
\newpage
\subsection*{Integrante 2: Javier Gómez}

\subsubsection*{Criterios Seleccionados}
\begin{itemize}
    \item Criterio 3: Confianza en IA
    \item Criterio 4: Frecuencia vs Confianza
\end{itemize}

\subsubsection*{Justificación: }

\begin{itemize}
    \item El índice de confianza en la IA es importante para comprender cómo el público percibe las tecnologías de éstas mismas, una mayor confianza se asocia con mayores tasas de aprobación de la IA. Además, 
    comprender estos niveles puede ayudar a identificar áreas de mejora en el diseño e implementación de la IA, como la transparencia y la seguridad. 
    Esta información puede ayudar a segmentar a los usuarios según su nivel de confianza y a desarrollar estrategias personalizadas para impulsar la adopción en el uso.
    \item Por otro lado, el criterio de frecuencia versus confianza ayuda a determinar si el uso frecuente de la IA genera una mayor confianza en ella o si, a pesar de su uso constante, los usuarios siguen siendo cautelosos. 
    Si la confianza no aumenta con la frecuencia de uso, puede ser una señal de que los usuarios necesitan más educación o apoyo.
\end{itemize}


\subsubsection*{Gráfico 3: }
\begin{figure}[H]
    \centering
    \includegraphics[width=0.85\textwidth]{Graficos/beeswarn.png}
    \caption[3]{Fuente: Elaboración propia con datos}
\end{figure}

\newpage
\textbf{Conclusión:}  
El gráfico sugiere que, en este grupo de 30 personas, la confianza en las respuestas de IA es variada, con posible predominio de posturas moderadas. Esto refleja la complejidad de un tema aún en evolución, 
donde la percepción depende de múltiples factores individuales y contextuales. Seria interesante extrapolar la investigación hacia grupos mas diferenciables para determinar si la edad u profesión podrían llegar a ser un
factor que determine el nivel de confianza que se tenga en las respuestas otorgadas por la IA, por ejemplo los datos recolectados fueron principalmente universitarios, el cual la mayoría presenta un contexto similar.

\subsubsection*{Gráfico 4: }
\begin{figure}[H]
    \centering
    \includegraphics[width=0.85\textwidth]{Graficos/spin plot.png}
    \caption[4]{Fuente: Elaboración propia con datos}

\end{figure}


\textbf{Conclusión:} 
\begin{itemize}
    \item Los valores más altos podrían corresponder a \textbf{Alta}, pero no a \textbf{Muy alta}, lo que refleja un escepticismo generalizado incluso entre usuarios frecuentes.

    \item El gráfico sugiere que la frecuencia del uso de IA aumenta con la confianza que se tiene a esta misma, pero persisten reservas incluso entre usuarios habituales. Esto subraya la necesidad de educación tecnológica y diseño de sistemas de IA más transparentes para cerrar la brecha de confianza.
\end{itemize} 


% ===================== MATÍAS ELGUETA =====================
\subsection*{Integrante 3: Matías Elgueta}

\subsubsection*{Criterios Seleccionados}
\begin{itemize}
    \item Criterio 1: 
    \item Criterio 2: 
\end{itemize}

\subsubsection*{Justificación:}

\begin{itemize}
    \item Diseñar campañas de concientización específicas por grupo etario, enfocándose en los públicos que muestran menor preocupación o menor conocimiento sobre el uso de sus datos.
    \item Adaptar políticas de privacidad y términos de uso en plataformas digitales, considerando el nivel de confianza que cada grupo etario tiene respecto al tratamiento de sus datos personales.
\end{itemize}

\subsubsection*{Justificación: Penetracion 5G}
Nos permite visualizar la evolución para la penetración de la red 5G entre 2015 y 2025 en distintos países, lo que facilita la comparación del ritmo de adopción tecnológica a nivel internacional y revela posibles brechas digitales, lo que puede ser útil para:
\begin{itemize}
    \item Tomar decisiones de inversión en infraestructura tecnológica, identificando países con mayor crecimiento 5G, lo cual puede orientar estrategias de empresas del sector de telecomunicaciones.
    \item Diseñar políticas públicas o marcos regulatorios que impulsen una adopción más equitativa de tecnologías avanzadas, especialmente en países con baja penetración, fomentando así la inclusión digital.
\end{itemize}

\subsubsection*{Gráfico 4: Preocupación ante el uso no autorizado de datos personales por rango etario.}
\begin{figure}[H]
    \centering
    \includegraphics[width=0.85\textwidth]{Graficos/spin plot.png}
    \caption[5]{Fuente: Elaboración propia con datos}
\end{figure}

\textbf{Conclusión:}  
\begin{itemize}
    \item Se puede observar una preocupación alta en todos los grupos presentando una mediana cercana a 4 a través de todos los rangos etarios.
    \item Al observar detalladamente cada violín se puede notar que los dos últimos rangos etarios (36-45 y 46+) poseen un valor mínimo de 1, lo que implica que existen encuestados pertenecientes a esos grupos que demuestran la preocupación mínima de 1 respecto al uso de sus datos personales, este no es el caso para los grupos más jóvenes (18-25 y 26-35) donde se observa que incluso los encuestados con menor preocupación poseen por lo menos un nivel ligero de esta.
    \item Junto con esto si nos centramos en los dos últimos grupos etarios, observando cada área, podemos ver que el grupo con edades de 36-45 años son el grupo con la menor preocupación, mostrando también la menor media indicada por la tercera línea horizontal.
\end{itemize}

\subsubsection*{Gráfico 4: Penetracion 5G por país.}
\begin{figure}[H]
    \centering
    \includegraphics[width=0.85\textwidth]{Graficos/spin plot.png}
    \caption[6]{Fuente: Elaboración propia con datos}

\end{figure}


\textbf{Conclusión:}  
\begin{itemize}
    \item Se observa que las curvas suben y bajan visiblemente para casi todos los países, en lugar de crecer de forma sostenida. Esto indica que la penetración 5G ha tenido retrocesos interanuales, lo cual puede deberse a factores técnicos, económicos o políticos que afectan el mantenimiento o la expansión de la red.
    \item Visualmente, países como Francia, Alemania y Japón presentan curvas más suaves y planas, lo cual sugiere un despliegue más progresivo y controlado en comparación con otros países donde hay caídas mas bruscas.

\end{itemize}

% ---------------------------------------------------------------------------------}
\section{Evidencia de encuesta aplicada}

A continuación, se presenta una imagen del archivo en Excel que contiene las respuestas recolectadas de la encuesta sobre el uso de inteligencia artificial. Esta tabla incluye los datos utilizados para generar los gráficos presentados anteriormente. Se realizo en google forms para mayor comodidad y cantidad de respuestas. 

\begin{figure}[H]
    \centering
    \includegraphics[width=0.95\textwidth]{Graficos/excel.png} % <-- nombre del archivo imagen
    \caption{Captura de pantalla del archivo Excel con las respuestas de la encuesta}
\end{figure}

El archivo completo puede ser consultado en el siguiente enlace:

\begin{itemize}
    \item \href{https://docs.google.com/spreadsheets/d/1KRdGL7pflDiA8TAT_pvPSt6OpXGPv6h4NTGceC8xDS0/edit?resourcekey=&gid=168064728#gid=168064728}{Ver archivo Excel de la encuesta}
\end{itemize}


\textbf{Repositorio:}  
\label{anexo:repositorio}

Acceso al repositorio en el siguiente link: 
\url{https://github.com/soloimsad/VD.git}

\end{document}
